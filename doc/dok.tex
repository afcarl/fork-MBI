% Preambuła
\documentclass[a4paper,10pt]{article}
\usepackage[polish]{babel}
\usepackage[OT4]{fontenc}
\usepackage[utf8]{inputenc}

% Część główna

\title{Metody bioinformatyki (MBI)\\Dokumentacja wstępna projektu}
\author{Michał Andrzejczak\and Karol Piczak\and	Andrzej Smyk}
\begin{document}
	\maketitle
	\section{Temat projektu}

	Dopasowywanie sekwencji za pomocą algorytmów Gotoha oraz \linebreak\mbox{Altschula - Ericksona}. Porównanie z inną dostępną implementacją (np. z pakietem Biostrings w języku R).

	\section{Cel projektu}

	Podstawowym zadaniem jest implementacja dwóch algorytmów służących do badania podobieństwa dwóch sekwencji np. nukleotydów: algorytmu Gotoha oraz \mbox{Altschula - Ericksona}. Oba korzystają z afinicznej funkcji kary za przerwę. 

	Następnie, za pomocą zaimplementowanych algorytmów przeprowadzonych zostanie szereg prób uliniowienia dla sekwencji o różnych długościach. Dla każdej próby zostanie zmierzony czas w jakim dopasowanie zostało ukończone. Na tej podstawie będziemy w stanie empirycznie zweryfikować zależność złożoności obliczeniowej algorytmów względem długości dopasowywanych sekwencji. 

	Pomiary czasów wykonania zostaną powtórzone dla jednego z istniejących pakietów umożliwiających dopasowywanie sekwencji parami, np. pakietu \mbox{Biopython}~\cite{biopython} lub Biostrings~\cite{biostrings}. Pozwoli to na porównanie względnej złożoności czasowej naszej implementacji z rozwiązaniami gotowymi. 

	\section{Funkcjonalność rozwiązania}

	Projekt zostanie zrealizowany jako aplikacja konsolowa lub okienkowa z prostym GUI. Podstawowa funkcjonalność jaka powinna się w nim znaleźć to:
	\begin{itemize}
		\item Wczytanie oraz sparsowanie danych z pliku tekstowego lub podanych bezpośrednio przez użytkownika: program pobierze dwie sekwencje nukleotydów oraz opcjonalnie wartości do systemu punktacji dla identycznych pozycji, różnych pozycji oraz kar za rozpoczęcie i kontynuację przerwy.
		\item Uliniowienie obu sekwencji za pomocą wybranego przez użytkownika algorytmu: wyznaczenie macierzy dopasowania oraz na ich podstawie najbardziej optymalnego dopasowania. 
		\item Wydrukowanie na standardowe wyjście lub zapis do pliku wyniku dopasowania, z oznaczeniem przerw oraz zgodnych i niezgodnych par nukleotydów z obu sekwencji.
	\end{itemize}

	\section{Dopasowanie par sekwencji}

	Dopasowanie pary sekwencji wykonuje się przy założeniu, że są one homologiczne, tj. wyewoluowały od jednego przodka. Różnice na poszczególnych pozycjach w obu sekwencjach mogą być wynikiem mutacji. Przerwy w sekwencjach sugerują zajście delecji lub insercji w jednej lub w obu sekwencjach.

	Afiniczna funkcja kary za przerwę jest najczęściej wykorzystywanym sposobem punktacji przerw. Za je stosowaniem przemawia fakt, iż insercje i delecje mają charakter blokowy, tj. istnieje większe prawdopodobieństwo wystąpienia jednej dużej przerwy, niż kilku mniejszych, znajdujących się nieodległym sąsiedztwie.Kara za utworzenie nowej przerwy jest równa $g_{open}$, zaś kara za jednostkowe wydłużenie istniejącej przerwy wynosi $g_{ext}$. Całkowita kara za przerwę o długości $l$ wynosi \mbox{$W(l) = g_{open} + g_{ext}(l - 1)$}

	Algorytm Gotoha~\cite{gotoh} pozwala na znalezienie jednego, optymalnego dopasowania  pomiędzy dwiema sekwencjami. Algorytm Altschula - Ericksona\cite{ae}, który pozwala na znalezienie wszystkich optymalnych sekwencji, jest w rzeczywistości modyfikacją algorytmu Gotoha. Właściwa różnica pomiędzy nimi dwoma, polega na różnych sposobach zapamiętywania ''ścieżek'', pozwalających później na wyznaczenie najbardziej optymalnego dopasowania (lub wszystkich optymalnych dopasowań, jak w algorytmie A--E).

	W obu algorytmach konieczne jest utworzenie w sumie 3 macierzy kosztu:
	\begin{description}
	\item[$M$]
	 - elementy $M_{i,j}$ tej macierzy, stanowią ocenę najlepszego dopasowania kończącego się na pozycjach $i$ i $j$ odpowiednio pierwszej sekwencji $a$ i drugiej sekwencji $b$,
	\item[$I$]
	 - macierz najlepszego dopasowania zakończonego delecją (dopasowanie kończy się na $a_i$ w pierwszej sekwencji oraz znakiem przerwy w drugiej),
	\item[$J$]
	 - macierz najlepszego dopasowania zakończonego insercją (dopasowanie kończy się znakiem przerwy w pierwszej sekwencji oraz resztą $b_j$ w drugiej).
	\end{description}
	Zależności rekurencyjne dla wymienionych powyżej macierzy można przedstawić następująco:
	\[
		M_{i,j} = max \left\{
			\begin{array}{ll}
				M_{i-1, j-1} + S(a_i, b_j)\\
				I_{i, }\\
				J_{i, }\\
			\end{array} \right.
	\]\\
	\[
		I_{i,j} = max \left\{
			\begin{array}{ll}
				M_{i-1, j} - g_{open}\\
				I_{i-1, j} - g_{ext}\\
			\end{array} \right.
	\]\\
	\[
		J_{i,j} = max \left\{
			\begin{array}{ll}
				M_{i, j-1} - g_{open}\\
				J_{i, j-1} - g_{ext}\\
			\end{array} \right.
	\]

	
	Oda algorytmy, zaimplementowane przy wykorzystaniu programowania dynamicznego, mają złożoność obliczeniową $O(mn)$, czyli taką samą jak ich odpowiedniki liniową funkcją kary za przerwę. Ponieważ wykorzystują one dodatkowo dwie macierze $I$ oraz $J$, złożoność pamięciowa jest trzykrotnie wyższa w porównaniu do algorytmów stosujących liniową funkcję kary.

	\section{Testowanie}

	Testowanie gotowego rozwiązania będzie składało się z dwóch części:
	\begin{enumerate}
		\item Testów wydajnościowych: ich celem będzie sprawdzenie czasów wykonania obu algorytmów dla sekwencji o różnej długości. Uzyskane czasy pozwolą na empiryczne oszacowanie złożoności czasowej algorytmów oraz porównanie ich z czasami wykonania dostępnych implementacji.
		\item Testów poprawnościowych: ich celem będzie weryfikacja, czy oba algorytmy prawidłowo dopasowują obie sekwencje tj. wykrywają zarówno mutacje, jak i insercje oraz delecje. W tym celu przetestujemy ich działanie na parach sekwencji, w których druga będzie zmodyfikowaną o mutacje oraz delecje i insercje wersją pierwszej.
	\end{enumerate}

	\section{Odnośniki}

	\begin{thebibliography}{99}
		\bibitem{biopython} http://biopython.org
		\bibitem{biostrings} http://master.bioconductor.org/packages/release/bioc/html/Biostrings.html
		\bibitem{gotoh} http://www.cs.unibo.it/~dilena/LabBII/Papers/AffineGaps.pdf
		\bibitem{ae} http://link.springer.com/article/10.1007\%2FBF02462326
	\end{thebibliography}
\end{document}
