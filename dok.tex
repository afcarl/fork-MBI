% Preambuła
\documentclass[a4paper,10pt]{article}
\usepackage[polish]{babel}
\usepackage[OT4]{fontenc}
\usepackage[utf8]{inputenc}

% Część główna

\title{Metody bioinformatyki (MBI)\\Dokumentacja wstępna projektu}
\author{Michał Andrzejczak
		\and
		Karol Piczak
		\and
		Andrzej Smyk
		}
\begin{document}
	\maketitle
	\section{Temat projektu}

	\indent Dopasowywanie sekwencji za pomocą algorytmów Gotoha oraz \linebreak\mbox{Altschula - Ericksona}. Porównanie z inną dostępną implementacją (np. z pakietem Biostrings w języku R)

	\section{Dopasowanie pary sekwencji}

	Dopasowanie pary sekwencji wykonuje się przy założeniu, że są one homologiczne, tj. wyewoluowały od jednego przodka. Różnice w sekwencjach mogą być wynikiem mutacji (która mogła zajść zarówno w pierwszej, drugiej lub obu sekwencjach). Przerwy w sekwencjach sugerują zajście delecji lub insercji w jednej lub obu sekwencjach.

	Afiniczna funkcja kary za przerwę jest najczęściej wykorzystywanym sposobem punktacji przerw. Kara za utworzenie nowej przerwy jest równa $g_{open}$, zaś kara za jednostkowe wydłużenie istniejącej przerwy wynosis $g_{ext}$. W tym wypadku całkowita kara za przerwę o długości $l$ wynosi \mbox{$W(l) = g_{open} + g_{ext}(l - 1)$}

	Całkowita ocena dopasowania sekwenci wynosi:
	\[\sum S(\alpha, \beta) - \sum W(l) \]

	Implementacja algorytmu dopasowywania sekwencji z afiniczną karą za przerwę ma złożoność obliczeniową $O(mn)$, czyli taką samą jak algorytmy dopasowania globalnego i lokalanego z liniową funkcją kary za przerwę.

	Za stosowanie afinicznej kary za przerwę przemawia fakt, że insercje i delecje mają charakter blokowy, tj. jest większe prawdopodobieństwo wystąpienia jednej dużej przerwy, niż kilku mniejszych, znajdujących się nieodległym sąsiedztwie.

\end{document}
