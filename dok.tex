% Preambuła
\documentclass[a4paper,10pt]{article}
\usepackage[polish]{babel}
\usepackage[OT4]{fontenc}
\usepackage[utf8]{inputenc}

% Część główna

\title{Metody bioinformatyki (MBI)\\Dokumentacja wstępna projektu}
\author{Michał Andrzejczak
		\and
		Karol Piczak
		\and
		Andrzej Smyk
		}
\begin{document}
	\maketitle
	\section{Temat projektu}

	Dopasowywanie sekwencji za pomocą algorytmów Gotoha oraz Altschula-Ericksona. Porównanie z inną dostępną implementacją (np. z pakietem Biostrings w języku R)

	\section{Dopasowanie pary sekwencji}

	Dopasowanie pary sekwencji wykonuje się przy założeniu, że są one homologiczne, tj. wyewoluowały od jednego przodka. Różnice w sekwencjach mogą być wynikiem mutacji (która mogła zajść zarówno w pierwszej, drugiej lub obu sekwencjach). Przerwy w sekwencjach sugerują zajście delecji lub insercji w jednej lub obu sekwencjach.

\end{document}
